\beginsong{Tinneke van Heule}
\beginverse*
Tinneke van Heule, ons maartje,
kan werken gelijk een paardje,
kan melken, kan mesten,
kan schuren gelijk de beste.
Tinneke van Heule, ons maartje,
staat hoog in de gunst van mijn vaartje
en als moederken haar prijst, dat mijn zuster er om krijst,
dan lach ik een beetj’ in mijn baardje.
\endverse
\beginchorus
Liever dan een vis die in de goudzee zwemt,
liever dan een vogel die geen sparen kent,
liever dan een fraule, Tinneke van Heule
Tinneke, ons maartje, in haar hemd. \rep{2}
\endchorus
\beginverse*
Tinneke heeft geld noch goedje, 
noch landeke, noch panneke, noch koetje,
noch huisje, noch kruisje,
noch een lappeke voor op mijn buisje.
Tinneke heeft geld noch goedje,
maar een hemel in haar lachen en haar grotje,
als zij trippelt naar de bron, met haar emmer in de zon,
en haar klompeken vast aan haar voetje.
\endverse
\beginverse*
Tinneke van Heule, mijn minneken,
op U staat mijn zoetste zinneken.
U lust ik, U kust’ ik, 
op uw harteken bouw en rust ik.
Tinneke van Heule, mijn minneken,
mijn poezelig dubbel kinneken,
leg uw handeken in de mijn, en een bruiloft zal het zijn,
van een boer en een schoon boerinneken.
\endverse
\endsong 