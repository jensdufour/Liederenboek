\beginsong{Klokke Roeland}
\beginverse*
Boven Gent rijst, eenzaam en grijs,
't Oud Belford, zinbeeld van 't verleden,
Somber en groots, steeds stom en doods,
Treurt d’ oude reus op 't Gent van heden,
Maar soms hij rilt en eensklaps gilt,
Zijn bronzen stemme door de stede.
Trilt in uw graf, trilt, Gentse helde,
Gij Jan Ryoens, Gij Artevelden,
Mijn naam is Roeland, 'k kleppe brand,
en luide storm in Vlaanderland.
\endverse
\beginverse*
En bont verschiet, Schept 't bronzen lied,
Prachtig weertov’rend mij voor d’ogen.
Mijn ziel herkent, Het oude Gent,
't volk komt gewapend toegevlogen:
't land is in nood “vrijheid of dood”,
De gilden komen aangetogen.
'k zie Jan Hyoens, 'k zie d’Artevelden,
en stormend roept Roeland de helden.
Mijn naam is Roeland, 'k kleppe brand,
En luide storm in Vlaanderland.
\endverse
\beginverse*
O heldentolk, O reuzenvolk,
O pracht en macht van vroeger dagen,
O bronzen lied, 'k Wete uw bedied,
En ik versta 't verwijtend klagen,
Doch wees getroost: zie 't Oosten bloost,
En vlaanderens zonne gaat aan 't dagen
“Vlaanderen die leu”, tril, oude toren,
En paar een lied met onze koren;
Zing “Ik ben Roeland, 'k kleppe brand,
Luide triomf in Vlaanderenland!”.
\endverse
\endsong