\beginsong{Den tandem}
\beginverse*
'k Hem ver ma liefken nen tandem gekocht,
't Es plezierig rijden en we leggen ons in den bocht. 
We rijden naar den buiten, ver van de stad,
Weg van de fabrieken, en recht in 't boerengat.
\endverse
\beginverse*
Eens daar aangekomen, op de groene wei,
De bloemekes die bloeien, de boerkes die zijn blij.
Dan fluistert z’in mijn oor: kom eens dichterbij,
Geef me ne kus, geeft er mij twee, geeft er mij drei. 
\endverse
\beginverse*
We eten boerenbrood, met witte en zwarte trippen,
't Es potverdekke ver an kinne van af te likken,
We spoelen alles deure, met grote potte bier,
Geeft er mij twee, geeft er mij drei, geeft er mij vier. 
\endverse
\beginverse*
We slapen op de schelf, onder 't warme stro,
Ik en mijn liefken en heure velo,
We vragen veur nen nacht te blijven on den boer zijn wijf,
Ze zeit: blijft er mor drei, blijft er mor vier, blijft er mor vijf. 
\endverse
\beginverse*
Mor eens is 'n tijd gekomen, om naar huis te gaan,
We komen van ons schelfken en we doen ons kleren aan,
En terwijl we ons veloken gaan pakken, zei ze tegen mij,
Zouden we ons broeiken hier nie bakken?
\endverse
\beginverse*
'k Hem ver mij liefken een schelfken gekocht,
't Es plezierig vrijen en we leggen ons in den bocht,
En hier op den buiten, ver van de stad,
Spelen al onze kindjes buiten \rep{2}
Met witte en zwarte trippen  \rep{2}
En veur ons deur ligt er een mat. 
\endverse
\endsong