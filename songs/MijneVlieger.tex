\beginsong{Mijne vlieger}
\beginverse*
'k Ben nie al te zot van 't spel, mor 'k vange gere musschen. 
Marblen en toppen kan ik wel, daarin ben ik nie fel.
'k Zie tegenwoordig overal, en ook al in mijn straatje,
Jongens schuppen op nen bal, mo 'k spele 't liefst van al
\endverse
\beginchorus
Mee mijnen vlieger, en zijne steert,
Hij goot omhuuge, 't es 't ziene weert,
'k geve mor klauwe, op mijn gemak,
'k hem nog drei bollekes, in mijnen zak. 
\endchorus
\beginverse*
Mietje van de koolmarchang, een meisken uit mijn strotjen,
Keurde mijne cervolant, en z’had er 't handje van,
Want zo rap als de wind was z’aan 't spelen met mijn klauwe,
En ze riep 't es 't spelen weert, want hij ee ne goeie steert. 
Ja mijne vlieger … twee bollekes … 
\endverse
\beginverse*
't Seef liet zijne vlieger op van 't soepe, 't soepe, 't soepe,
Maar hij stuikt op zijne kop, en muile dat hij trok. 
Zij spankoorde was veel te kort, en met zijn 't sietse klauwe, 
En daarbij was zijne steert gen chique toebak weert.
Maar mijne vlieger … één bolleken …
\endverse
\beginverse*
Laatst op 't Sint-Denijsplein, mijne vlieger was aan’t zweve,
d’er kwam een wijf, een groot venijn, ze zei dat mag niet zijn. 
Hij hangt te veel in mijne weg, en ze begost er aan te sleuren,
En op een twee drei pardaf, de koorde schoot er af…
\endverse
\beginchorus
Hij was go vliegen, al mee de wind,
'k Stonde te schrieme, 'k was mor e kind,
Mijnen bol klauwe, die ging ne gang,
Dat zal 'k omtauwe, mijn leven lang…  
\endchorus
\endsong 