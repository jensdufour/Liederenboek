\beginsong{In München steht ein hofbräuhaus}
\beginverse*
Da wo die grüne Isar flüsst,
wo man mit Grüss Gott dich grüsst,
liegt meine schöne Münchnerstadt,
die ihres gleichen nicht had.
Wasser ist billig, frisch und gut,
nur verdünnt es unseres Blut.
Schöner sind tropfen goldnen weins,
aber am schönsten ist eins!
In München steht ein Hofbräuhaus,
eins, zwei, suffa.
Da läuft so manches Fässchen aus,
eins, zwei, suffa.
Da hat so mancher brave Mann,
eins, zwei, suffa,
gezigt was er schon vertragen kan.
Schon früh am Morgen fing er an,
und spät am Abend kam er nach Haus,
so schön ist’s im Hofbräuhaus!
Da trinkt man Bier nicht auf den Glas,
da gibt’s nur die grosse Hass,
und wen die erste Mass ist leer,
dan bringt dir die resl bald mehr.
Dann krigt zu Haus die Frau’nen schrek,
bleibt der Mann noch lange weg?
Aber die braven Nachbarleut,
die wissen besser bescheid!
\endverse
\endsong